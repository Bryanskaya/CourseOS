\section{Технологическая часть}

\subsection{Выбор языка программирования}
В качестве языка программирования был выбран С. \cite{c} Для сборки модуля использовалась утилита make.

Была выбрана среда разработки Visual Studio Code \cite{VStudio}, так как она бесплатная, кроссплатформенная,  а также позволяет использовать все возможности консоли, не переключаясь между окнами. \newline

\subsection{Структура модуля}
Реализованный модуль включает в себя следующие функции:
\begin{itemize}
	\item \textbf{fw\_init()} -- функция инициализации модуля;
	
	\item \textbf{fw\_exit()}-- функция выгрузки модуля;
	
	\item \textbf{hide()} -- функция изменения видимости модуля (скрытие);
	
	\item \textbf{unhide()} -- функция изменения видимости модуля (обнаружение);
	
	\item \textbf{fw\_read(struct file *filp, char \_\_user *buff, size\_t count, loff\_t *f\_pos)} -- функция чтения, описываемая в структуре struct file\_operations;
	
	\item \textbf{fw\_write(struct file *filp, const char \_\_user *buff, size\_t count, loff\_t *f\_pos)} -- функция записи, описываемая в структуре struct file\_operations;
	
	\item \textbf{add\_rule(struct fw\_rule *rule)} -- добавление нового правила;
	
	\item \textbf{del\_rule(struct fw\_rule *rule)} -- удаление правила;
	
	\item \textbf{fw\_in\_filter(void *priv, struct sk\_buff *skb, const struct nf\_hook\_state *state)} -- <<обёртка>> функции фильтрации для входящих пакетов;
	
	\item \textbf{fw\_out\_filter(void *priv, struct sk\_buff *skb, const struct nf\_hook\_state *state)} -- <<обёртка>> функции фильтрации для исходящих пакетов;
	
	\item \textbf{filter(void *priv, struct sk\_buff *skb, const struct nf\_hook\_state *state,
	struct list\_head *list\_rule)} -- основная функция фильтрации пакетов;
	
	\item \textbf{str\_rule(struct fw\_rule *rule)} -- функция преобразования правила фильтрации в удобный для восприятия человеком вид;
	
	\item \textbf{str\_packet(uint32\_t src\_ip, uint16\_t src\_port,uint32\_t dest\_ip, uint16\_t \, dest\_port, char *protocol\_str)} -- функция преобразования информации о перехваченном пакете в удобный для восприятия человеком вид. \newline
\end{itemize}
%
Были также определены структуры:
%
\begin{itemize}
	\item struct file\_operations;
	
	\item struct miscdevice;
	
	\item struct nf\_hook\_ops. \newline
\end{itemize}
%
В Приложении А представлены листинги кода программы. \newline
%
\subsection{Makefile}

\subsection{Демонстрация работы модуля}

\subsection{Вывод}