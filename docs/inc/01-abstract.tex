\section*{ВВЕДЕНИЕ}
\addcontentsline{toc}{section}{ВВЕДЕНИЕ}

Информация -- это один из важнейших ресурсов, который представляет из себя движущую силу развития человечества. Потребность в ней -- одна из основных для современного человека. Б\textbf{о}льший процент знаний приходится на приобретённые либо вследствие непосредственного получения информации, либо в результате анализа уже существующих данных. 

Объём информационных ресурсов в любой области растёт огромными темпами, это связано, прежде всего, с усложнением всех сфер жизнедеятельности общества. Кроме того, непрерывно увеличиваются массивы передаваемой информации, речь идёт не только о бытовых разговорах, но и всего инфопотока в Интернете в целом. 

Технический прогресс не стоит на месте, и сейчас практически каждый компьютер подключается к сети для обмена какими-либо данными. Компьютерная сеть изначально является незащищённой и уязвимой для внешних атак системой. И для того, чтобы предотвратить несанкционированный доступ к устройству, подключенному к глобальной или частной сети, необходимо использовать специальные программные средства, называемые \textbf{межсетевыми экранами} (также известные, как сетевые фильтры, брандмауэры, Firewall-ы).
