\section*{ВВЕДЕНИЕ}
\addcontentsline{toc}{section}{ВВЕДЕНИЕ}

Информация -- это один из важнейших ресурсов, который представляет из себя движущую силу развития человечества. Потребность в ней -- одна из основных для современного человека. Б\textbf{о}льший процент знаний приходится на приобретённые либо вследствие непосредственного получения информации, либо в результате анализа уже существующих данных. 

Объём информационных ресурсов в любой области растёт огромными темпами, это связано, прежде всего, с усложнением всех сфер жизнедеятельности общества. Кроме того, непрерывно увеличиваются массивы передаваемой информации, речь идёт не только о бытовых разговорах, но и всего инфопотока в Интернете в целом. 

Технический прогресс не стоит на месте, и сейчас практически каждый компьютер подключается к сети для обмена какими-либо данными. Компьютерная сеть изначально является незащищённой и уязвимой для внешних атак системой. И для того, чтобы предотвратить несанкционированный доступ к устройству, подключенному к глобальной или частной сети, необходимо использовать специальные программные средства, называемые межсетевыми экранами (также известные, как сетевые фильтры, брандмауэры, Firewall-ы).

\underline{Цель данной работы} - разработать межсетевой экран, осуществляющий контроль проходящего через него сетевого трафика, в виде загружаемого модуля.

Необходимо предоставить пользователю возможность задания правил фильтрации и изменения видимости модуля в системе.

Для достижения поставленной цели необходимо решить следующие задачи:
\begin{enumerate}
	\item изучить основные принципы работы сети и межсетевых экранов;
	
	\item ознакомиться со способом перехвата пакетов сети;
	
	\item проанализировать особенность misc драйвера и основные принципы работы с ним;
	
	\item изучить методы передачи информации из пространства пользователя в пространство ядра и наоборот;
	
	\item реализовать межсетевой экран.
\end{enumerate}
