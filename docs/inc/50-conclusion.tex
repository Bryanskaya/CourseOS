\section*{ЗАКЛЮЧЕНИЕ}
\addcontentsline{toc}{section}{ЗАКЛЮЧЕНИЕ}

В процессе выполнения данной работы были изучены основные принципы функционирования отдельных узлов распределённой системы и межсетевых экранов. Для реализации межсетевого экрана было выбрано misc устройство, поскольку оно направлено на выполнений одной конкретной задачи. Для фильтрации поступающих на хост пакетов были разработаны хук-функции, заданные в точках перехвата до маршрутизации и после неё, определена необходимая для их регистрации структура ядра struct nf\_hoop\_ops. В целях увеличения эффективности и безопасности работы загружаемого модуля реализована функция его сокрытия в системе с возможностью дальнейшего восстановления видимости. Разработан формат правил фильтрации, необходимых для работы межсетевого экрана, позволяющий анализировать пакеты не только по одному, но и по нескольким признакам одновременно.

Таким образом, был разработан загружаемый модуль ядра, выполняющий роль межсетевого экрана, осуществляющего контроль проходящего сетевого трафика в соответствии с правилами, которые задаёт и может корректировать пользователь. 

\pagebreak