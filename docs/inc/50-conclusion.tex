\section*{ЗАКЛЮЧЕНИЕ}
\addcontentsline{toc}{section}{ЗАКЛЮЧЕНИЕ}

В ходе работы были выполнены поставленная цель и необходимые для этого задачи:
\begin{itemize}
	\item были изучены основные принципы работы сети и межсетевых экранов;

	\item а также способы перехвата пакетов сети (основные параметры задаются в структуре struct nf\_hook\_ops) и описания правил;
	
	\item были изучены особенности misc драйвера:
	\begin{itemize}
		\item не нужно указывать старший номер устройства -- он задан заранее;
		
		\item младший номер следует выделять динамически;
		
		\item все необходимые структуры создаются автоматически при регистрации, в отличие от char драйвера, при работе с которым нужно вызывать функции cdev\_init, cdev\_add, class\_create;
	\end{itemize}

	\item основные принципы работы с подобным драйвером;
	
	\item проработан метод передачи информации из пространства пользователя в пространство ядра и наоборот с помощью функций copy\_from\_user и copy\_to\_user;
	
	\item разработано соответствующее программное обеспечение.\newline
\end{itemize}

Таким образом, был реализован загружаемый модуль ядра, выполняющий роль межсетевого экрана, осуществляющего контроль проходящего через него сетевого трафика в соответствии с правилами, которые задаёт пользователь. 

\pagebreak