\section*{ЗАКЛЮЧЕНИЕ}
\addcontentsline{toc}{section}{ЗАКЛЮЧЕНИЕ}

В процессе выполнения данной работы были изучены основные принципы функционирования отдельных узлов распределённой системы и межсетевых экранов. 

Для реализации межсетевого экрана было выбрано misc устройство, поскольку направлено на выполнение одной конкретной задачи и наиболее соответствует требованиям, выдвигаемых к задаче. 

Для фильтрации поступающих на хост пакетов были разработаны специальные хук-функции, заданные в точках перехвата до маршрутизации и после неё, в процессе реализации были определены необходимые для их регистрации структуры ядра struct nf\_hoop\_ops. 

В целях увеличения эффективности и безопасности работы загружаемого модуля реализована функция его сокрытия в системе с возможностью дальнейшего восстановления видимости. 

Были проработаны следующие признаки для фильтрации сетевых пакетов: по протоколу (TCP или UDP), по направлению (входящие или исходящие), по IP-адресу (источника или назначения), по порту и доменному имени аналогично. 

Разработан формат правил фильтрации, необходимых для работы межсетевого экрана, позволяющий анализировать пакеты не только по одному, но и по нескольким признакам одновременно.

Все заявленные возможности межсетевого экрана были успешно протестированы.

Таким образом, был разработан загружаемый модуль ядра, выполняющий роль межсетевого экрана, осуществляющего контроль проходящего сетевого трафика в соответствии с правилами, которые задаёт и может корректировать пользователь. 

\pagebreak